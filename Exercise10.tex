\documentclass[12pt]{article}
\usepackage{mathrsfs}
\usepackage{amsmath,amssymb}
\usepackage{graphicx}
\usepackage[utf8]{inputenc}
\usepackage{amsthm}
\usepackage{polynom}
\usepackage{tkz-berge}
\usepackage{pgf}
\usepackage{xcolor}
\usepackage{algorithm}
\usepackage{caption}
\usepackage{multicol}
\usepackage{array}
\usepackage[noend]{algpseudocode}
\usepackage{listings}
\newcommand{\Term}{Winter 2019}
\newcommand{\Course}{104.272 Discrete Mathematics, Group 1}

\newcommand{\Assignment}{10. Exercise}
\newcommand{\DueDate}{ 8 January, 2020 }

\usepackage[body={6in,9in}]{geometry}



\begin{document}
Hugo \textit{Rincon Galeana}
\begin{center}

\textbf{TU Wien, \Term} \\
\textbf{\Course} (Professor Gittenberger) \\
\textbf{\Assignment, Due \DueDate}
\end{center}


%%%%%%%%%%%%%%%%%%%%%%%%%%%%%%%%%%%%%%%%%%%%%%%%%%%%%%%%%%%%%%%%%%%%%%%%
%% *****      Type your answers after the "\soln" commands      ***** %%
%%%%%%%%%%%%%%%%%%%%%%%%%%%%%%%%%%%%%%%%%%%%%%%%%%%%%%%%%%%%%%%%%%%%%%%%
\begin{enumerate}
    \setcounter{enumi}{90}
    \item Let $(e,n)$ and $(d,n)$ be Bob's public and private RSA keys, respectively. Suppose that Bob sends an encrypted message $c$ and Alice wants to find out the original message $m$. She has the idea to send Bob a message and ask him to sign it. How can she find out $m$?
    
    \emph{Hint: Pick a random integer $r$ and consider the message $r^e c \textrm{ mod }n$. What property does $r$ need to satisfy?}
    
    \begin{proof}
    Notice that $c \underset{\textrm{mod }n}{\equiv} m^e $.
     First we apply the Euclidian algorithm in order to find $\textrm{gcd}(c,n)$. Notice that if $\textrm{gcd}(c,n) \neq 1$, then $\textrm{gcd} (c,n)$ is one of the prime factors, and we can obtain the private key.
     
    If $c$ is a unit in $\mathbb{Z}_n$, then we test $r^e c = r^e m^e$. Notice that if $r^e m^e \underset{\textrm{mod }n}{\equiv} 1$, then $r$ is the product inverse of $m$ in $\mathbb{Z}_n$. Therefore it suffices to invert $r$ modulo $n$ via the Euclidian algorithm in order to obtain $m$.
    
    Notice that this attack is impractical for a sufficiently large $n$ since it requires encrypting and multiplying for a large amount of numbers. Namely for $p \cdot q - (q+p-1)$.
    \end{proof}
    
    \item Let $G$ be a finite (abelian) group and $a \in G$ an element for which $\textrm{ord}_G(a)$ is maximal. Prove that for all $b\in G$, the order $\textrm{ord}_G(b)$ is a divisor of $\textrm{ord}_G(a)$.
    
    \begin{proof}
    Notice that it is important that $G$ is abelian, otherwise $S_3$ is an example of a non-abelian group in which the order of any 3-cycle is 3, the order of any 2-cycle is 2, and the identity permutation is of order 1. Therefore, an element of max order is of order $3$, however a 2-cycle is of order 2 which does not divide 3.
    
    Assume for a contradiction that $\textrm{ord}_G(b) \not | \; \textrm{ord}_G(a)$. From a previous theorem, there must exist $c \in G$ such that $\textrm{ord}_G(c) = \textrm{lcm}(\textrm{ord}_G(a), \textrm{ord}_G(b))$. Since $a$ is of maximal order in $G$, then $\textrm{ord}_G(a) \leq \textrm{lcm}(\textrm{ord}_G(a), \textrm{ord}_G(b)) \leq \textrm{ord}_G(a)$. This implies that $\textrm{ord}_G(a)$ is a multiple of $\textrm{ord}_G(b)$.
    
    This is a contradiction.
    \end{proof}
    
    \item Prove that if $G$ is a finite group and $a \in G$ is an element with $\textrm{ord}_G(a)=r$, then for every $k \in \mathbb{N}$, $\textrm{ord}_G(a^k) = r/ \textrm{gcd}(r,k)$.
    
    \begin{proof}
        Let $c:= a^k$. Notice that $c^{r/\textrm{gcd(r,k)}} = a^{k \cdot r / \textrm{gcd}(r,k)} = a^{lcm(r,k)} = e$.
        
        Therefore $\textrm{ord}_G(c) \: | \: r /\textrm{gcd}(r,k)$.
        
        Also since $\textrm{ord}_G(a^k):= m$ is the smallest positive integer such that $(a^k)^m = e$, then $a^{k \cdot m} = e$, then $k\cdot m$ is a multiple of $r$, in fact $k \cdot m = \textrm{lcm}(k,r)$, otherwise $\textrm{lcm}(k,r) = k \cdot m'$ with $m' < m$, thus $a^{k\cdot m'} = e$, therefore $c^{m'} = e$, contradicting that $\textrm{ord}_G(c) = m$.
        
        Now notice that $k \cdot m = \textrm{lcm}(k,r) = k \cdot r / \textrm{gcd}(k,r)$. Therefore $ m = r/\textrm{gcd}(k,r)$.
    \end{proof}
    
    \item Use the Euclidean algorithm to find all greatest common divisors of $x^3+ 5x^2+7x+3$ and $x^3+x^2-5x+3$ in $\mathbb{Q}[x]$.
    
    \begin{proof}
    
    \polylongdiv[style=B]{x^3+ 5x^2+7x+3}{x^3+x^2-5x+3}
    
    \polylongdiv[style=B]{x^3+x^2-5x+3}{x^2+3x}
    
    \polylongdiv[style=B]{x^2+3x}{x+3}
    
    This shows that $x+3= \textrm{gcd}(p(x),q(x))$ 
    \end{proof}
    
    \item Prove that $x^4 + x^3+1$ is irreducible over $\mathbb{Z}_2$.
    
    \begin{proof}
    Notice that evaluating $p(0) = 1 = p(1)$. Therefore $p(x)$ doesn't have factors of degree 1, and therefore no factors of degree 3. 
    
    Let us consider a polynomial $q(x)$ of degree 2. Notice also that in $\mathbb{Z}_2$ the following equation holds $x^2 = x$. Therefore any polynomial of degree 2 in $\mathbb{Z}_2, x^2 + a x + b$ has the same root as $(1+a) x + b$. Notice that if $b = 0$ then $q(x)$ has a root. Also if $b= 1$ and $a = 0$, then it also has a root. Therefore $q(x) = x^2 + x + 1$ is the only polynomial of degree 2 without a root in $\mathbb{Z}_2$.
    
    This means that $q(x)$ is the only viable candidate for being a factor of $p(x)$, since the rest of the polynomials imply the existence of at least one root. However $(x^2 +x +1)^2 = x^4 +x^3+ x^2+x^3+x^2+x+x^2+x+1 = x^4+x^2+1 \neq p(x)$.
    
    Therefore $p(x)$ is irreducible.
    \end{proof}
    
    \item List all irreducible polynomials up to degree 3 in $\mathbb{Z}_3$
    Since $\mathbb{Z}_3$ is a field, we will only consider irreducible polynomials with main coefficient 1.
    
    We start with the polynomials of degree 1
    
    $$ x, x+1, x+2 $$.
    
    We proceed with the polynomials of degree 2.
    
    Notice that a polynomial of degree 2 is irreducible if and only if it does not have a root. Notice that the independent term should be different than $0$ in order to avoid the root $0$.
    
    We test the polynomials and we obtain 
    
    $$ x^2 +1; x^2+x+1$$
    $$ x^2 +x+2; x^2+2x+2$$
    
    Now, for the polynomials of degree 3 we notice that in $\mathbb{Z}_3$, $x^3 = x$. Since a non reducible polynomial of degree 3 may always be factored by a polynomial of degree 1, it is sufficient to show that it doesn't have a root in order to show that it is irreducible.
    
    Also notice from the first observation that $p(x) = x^3 + a_2 x^2 + a_1 x + a_0$ is irreducible if and only if $a_2 x^2 + (a_1+1)x + a_0$ is irreducible.
    
    We now proceed to find all irreducible monic polynomials of degree 3 by using the list of monic polynomials of degree 2.
    
    
    $x^2 +1 $ yields $x^3 + x^2 +2x+1 $ and $x^3 + 2 x^2 + 2x + 2$
    
    $x^2+x+1$ yields $x^3 +x^2+1$ and $x^3+2x^2+x+2$
    
    $x^3+x^2+2$ yields $x^3+x^2+2$ and $x^3+2x^2+x+1$
    
    $x^2 +2x+2$ yields $x^3+x^2+x+2$ and $x^3+2x^2+1$
    
    \item Let $\mathbb{K}$ be a field and $p(x) \in \mathbb{K}[x]$ a polynomial of degree $m$. Prove that $p(x)$ cannot have more than $m$ zeroes (counted with multiplicities). \emph{Hint: Use the fact that $\mathbb{K}[x]$ is a factorial ring.}
    \begin{proof}
    Assume by contradiction that $p(x)$ has at least $m+1$ roots.
    
    Consider the following sequences of roots $(r_1, \ldots, r_m)$ and $(r_1, \ldots, r_{m-1},r_{m+1})$. Notice that the first sequence yields the factorization $(x-r_1) (x-r_2) \ldots (x-r_m)$ for $p(x)$. The second sequence yields the factorization $(x-r_1) \ldots (x-r_{m-1}) (x-r_{m+1})$. This contradicts the fact that $\mathbb{K}[x]$ is a factorial ring.
    \end{proof}
    
    \pagebreak
    \item Let $R$ be a ring and $(I_j)_{j \in J}$ be a family of ideals of $R$. Prove that $\displaystyle\bigcap \limits_{j \in J} I_j$ is also an ideal of $R$.
    
    \begin{proof}
     Notice that since each of the $I_j$ is an ideal, then $0 \in I_j, \forall j \in J$. Therefore $0 \in \displaystyle\bigcap \limits_{j \in J} I_j$. Thus $\displaystyle\bigcap \limits_{j \in J} I_j \neq \varnothing$.
     
     
     Let $x, y \in \displaystyle\bigcap \limits_{j \in J} I_j$. Let $I_j$ be an ideal of the family, it follows that $x,y \in I_j$, therefore $x-y \in I_j$ for all $j \in J$. Therefore $x-y \in \displaystyle\bigcap \limits_{j \in J} I_j$.
     
     Let $a \in R$ and $i \in \displaystyle\bigcap \limits_{j \in J} I_j$. Since each of the $I_j$ are ideals of $R$, then $a \cdot i \in I_j$. Therefore $a\cdot i \in \displaystyle\bigcap \limits_{j \in J} I_j$.
     
     Therefore $\displaystyle\bigcap \limits_{j \in J} I_j$ has all the properties of an ideal of $R$.
    \end{proof}
    
    \item Let $R$ be a ring and $I$ an ideal of $R$. Then $(R/I,+)$ is the factor group of $(R,+)$ over $(I,+)$. Define a multiplication on $R/I$ by 
    
    $$ (a+I) \cdot (b+I) := (ab)+I.$$
    
    Prove that this operation is well defined, i.e. that 
    
    $$ a + I = c+ I \wedge b+I = d+ I \Rightarrow (ab)+ I = (cd)+I.$$
    
    Furthermore, show that $(R/I,+,\cdot)$ is a ring.
    
    \begin{proof}
     Assume that $a + I = c+ I \wedge b+I = d+ I$.
     
     Then $a = c+ i_1$ for some $i_1 \in I$, $b = d+i_2$ for some $i_2 \in I$.
     
     Consider $ab = (c+ i_1) \cdot (d + i_2) = cd +c i_2 + d i_1 + i_1 i_2$. Since $i_1, i_2 \in I$, then $c i_2 , d i_1 , i_1 i_2 \in I$. Since $I$ is an ideal, it is also closed under sums, therefore $c i_2 + d i_1 + i_1 i_2 = i_3 \in I$. Therefore $ab = cd + i_3$ for some $i_3 \in I$. this shows that $ab + I = cd + I $. 
     
     Notice that the $0$ element is given by $I$, the $e$-element is given by $e + I$.
     
     Also notice that the additive inverse of $a+I$ is given by $-a+ I$.
     
     Consider $((a+I) (b+I)) c+ I = (ab +I) (c+I) = (ab)c + I = a (bc) + I = a +I ((b+I) (c+I) )$. This shows the associative property of the product.
     
     Now consider $(a+ I + b+ I) (c+I) = (a+b+ I) (c+ I) = (a+b)c + I = ac + bc + I = ac +I + bc + I$. This shows the distributive property of the product (analogous for the left distributivity). 
     
     
     
    \end{proof}
    
    \item Let $U = \{ \bar 0, \bar 2, \bar 4 \} \subseteq \mathbb{Z}_6$. Show that $U$ is an ideal of $(\mathbb{Z}_6,+, \cdot)$. Is it a subring as well? Does it have a 1-element?.
    
    \begin{proof}
    Notice that $U$ are the even numbers of $\mathbb{Z}_6$. Since 6 is also even, then for any $z \in \mathbb{Z}_6$ and any $u \in U$, $zu \in U$. Notice that $U$ is closed under addition and under additive inverse since $-\bar 0 = \bar 0$ and $- \bar 2 = \bar 4$. Therefore it is closed under subtraction. 
    
    Therefore it is an ideal of $\mathbb{Z}_6$. Since it is an ideal, it is also a subring. Notice that $\bar 4$ is the 1-element of $U$.
    \end{proof}
    
    \end{enumerate}
    
    
\end{document}