\documentclass[12pt]{article}
\usepackage{mathrsfs}
\usepackage{amsmath,amssymb}
\usepackage{graphicx}
\usepackage[utf8]{inputenc}
\usepackage{amsthm}
\usepackage{polynom}
\usepackage{tkz-berge}
\usepackage{pgf}
\usepackage{xcolor}
\usepackage{algorithm}
\usepackage{caption}
\usepackage{multicol}
\usepackage{array}
\usepackage[noend]{algpseudocode}
\usepackage{listings}
\newcommand{\Term}{Winter 2019}
\newcommand{\Course}{104.272 Discrete Mathematics, Group 1}

\newcommand{\Assignment}{11. Exercise}
\newcommand{\DueDate}{ 15 January, 2020 }

\usepackage[body={6in,9in}]{geometry}



\begin{document}
Hugo \textit{Rincon Galeana}
\begin{center}

\textbf{TU Wien, \Term} \\
\textbf{\Course} (Professor Gittenberger) \\
\textbf{\Assignment, Due \DueDate}
\end{center}


%%%%%%%%%%%%%%%%%%%%%%%%%%%%%%%%%%%%%%%%%%%%%%%%%%%%%%%%%%%%%%%%%%%%%%%%
%% *****      Type your answers after the "\soln" commands      ***** %%
%%%%%%%%%%%%%%%%%%%%%%%%%%%%%%%%%%%%%%%%%%%%%%%%%%%%%%%%%%%%%%%%%%%%%%%%
\begin{enumerate}
\setcounter{enumi}{110}
    \item Prove that for each prime $n$, we have $(n-1)! \underset{n}{\equiv} -1 $. Further show that this only holds when $n$ is prime.
    
    \begin{proof}
     Lec $r^2 \underset{n}{ \equiv} 1$. From the division algorithm, $r = \alpha p + c'$. With $0 \leq c' < p$. Since $c'= 0$ would imply that $r^2 \underset{n}{\equiv} 0$, then $c' \geq 1$. Therefore $r = \alpha p + 1 + c$. Therefore $r^2 = \alpha^2 p^2 +2 \alpha p (1+c) +1 +2c + c ^2 = \beta p + c (2+c) + 1$. Since $r^2 = \gamma p + 1$, and $c < p$ with $p$ prime, then $p \: | 2+c$ or $p \: | c$. Therefore $c  \underset{n}{\equiv} p-2$ or $c = 0$. This implies that $r \underset{n}{\equiv} p-1$ or $r \underset{n}{\equiv} 1$. Therefore in $\mathbb{Z}_p$ all elements that are not $1$ or $p-1$ are not self-inverse. This implies that $(n-1)! = \prod \limits _{z \in \mathbb{Z_p}^*} z = 1 \cdot (p-1) = -1$, since all other elements cancel out with their respective inverses.
     
     Now assume that $(n-1)! \underset{n}{\equiv} -1$. Since $((n-1)!)^2 = 1$, this implies that all $z \in \mathbb{Z}_n$ such that $z \neq 0$ have a product inverse. Therefore $n$ must be prime.
    \end{proof}
    
    \item Prove that every finite integral domain is a field
    
    \emph{Hint: One only has to show that if $R$ is a finite integral domain, then every non-zero element of $R$ is invertible. One starts as follows: let $a \in R, a \neq 0$, and $f: R \rightarrow R$ be the function defined by $f(x) = ax$. First prove that $f$ is injective, than notice that $f$ is a function from a finite set to itself.}
    
    \begin{proof}
        We take the hint and let $f$ as defined previously. Notice that $f$ is a group morphism with respect to the product. Notice that since $R$ is an integral domain, then $\textrm{ker} f = 0$. Therefore $f$ is injective. Since $R$ is finite, then $f$ is bijective. Therefore there exists an element $c$ such that $ac = 1$. This completes the proof that $R$ is a field.
    \end{proof}
    
    \item Let $\mathbb{K}$ be a field whose characteristic equals a prime number $p$. Prove the so-called \emph{freshman's dream}:
    $$ (a+b)^p = a^p + b^p \quad \textrm{for all } a,b \in \mathbb{K}.$$
    
    Does this statement generalises for more than two summands?
    
    \begin{proof}
    Recall from the binomial theorem that: 
    
    $(a+b)^p = \displaystyle \sum \limits_{i=0}^p {p \choose i} a^{p-i}b^i$, also notice that $ \displaystyle p \: \vert {p \choose i}$ for all $0 < i < p$. Since $\mathbb{K}$ is of characteristic $p$ it follows that $\displaystyle{p \choose i} = 0 $ for all $0<i<p$.
    
    Notice that it works for any ammount of summands, that is $\displaystyle \left (\sum \limits_{i=1}^n a_i \right)^p = \displaystyle\sum \limits_{i=1}^n a_i^p $. The proof is induction over the number of summands.
    \end{proof}
    
    \item Prove that for each odd prime $p$, there is a field with $p^2$ elements.
    
    \emph{Hint: One has to show that there is an irreducible quadratic polynomial over $\mathbb{Z}_p$, for example of the form $x^2 -a$. To that end, one could show that there exists $a \in \mathbb{Z}_p$ which is not the square of another element in $\mathbb{Z}_p.$}
    
    \begin{proof}
    Consider the function $f: \mathbb{Z}_p \rightarrow \mathbb{Z}_p$, $x \mapsto x^2$. Since we are considering $p > 2$, then $f$ is not injective since $1^2 = 1 = (p-1)^2$. Since $\mathbb{Z}_p$ is also finite, then $f$ is not surjective. Therefore, there must exist $a \in \mathbb{Z}_p$ without a square root in $\mathbb{Z}_p$. This implies that $x^2 - a$ is irreducible in $\mathbb{Z}_p [x]$. Notice that $\mathbb{Z}_p[x] / x^2 -a$ is a Galois field with $p^2$ elements.
    \end{proof}
    
    \item Consider the field $\mathbb{Z}_2[x]/m(x) =\mathbb{F}_{256}$ where $m(x) = x^8+x^4+x^3+x+1$. Hence the residue classes modulo $m(x)$ are $\overline{b(x)} = \overline{b_7 x^7 + b_6 x^6 + \ldots b_1x + b_0}$ and can be identified with the byte $b_7b_6 \ldots b_1 b_0.$
    
    \begin{enumerate}
        \item Compute the sum and the product of the two bytes $10010101$ and $11001100$ in $\mathbb{F}_{256}.$
        \begin{proof}
        $$ \begin{array}{cr}
             &  10010101 \\
            + & 11001100\\
            \hline
            & 01011001 \\ 
        \end{array}$$
        
        Notice that multiplying by a fixed element is a linear transformation, therefore it can be defined through the image of the basis as a matrix product. Also notice that $x^8 = x^4 + x^3 +x + 1$ since in $\mathbb{Z}_2$ each element is its own additive inverse. Taking this into consideration, we can consider the product by $10010101$ as multiplying the following matrix $ M$ with a vertical vector
        
        $$\begin{array}{c c c c c c c c}
         1 & 0 & 0 & 1 & 1 & 0 & 0 & 1\\
         1 & 1 & 0 & 0 & 1 & 1 & 0 & 0\\
         1 & 1 & 1 & 0 & 0 & 1 & 1 & 0\\
         1 & 1 & 1 & 1 & 0 & 0 & 1 & 1\\
         0 & 1 & 1 & 0 & 0 & 0 & 0 & 0\\
         1 & 0 & 1 & 0 & 1 & 0 & 0 & 1\\
         0 & 1 & 0 & 1 & 0 & 1 & 0 & 0\\
         0 & 0 & 1 & 1 & 0 & 0 & 1 & 1\\
        \end{array}$$
        
        Which yields $00101000$
        \end{proof}
        \item Compute the multiplicative inverse $y^{-1}$ for $y = 10010101$ in $\mathbb{F}_{256}.$
        
        \begin{proof}
        We solve the equation $M \cdot y^{-1} = 00000001$ through the diagonalization method.
        
        Which yields the following equation $M' y^{-1} = 00000110$ where $M'$ is the following matrix :
        
        $$\begin{array}{c c c c c c c c}
         1 & 0 & 0 & 1 & 1 & 0 & 0 & 1\\
         0 & 1 & 0 & 1 & 0 & 1 & 0 & 1\\
         0 & 0 & 1 & 0 & 1 & 0 & 1 & 0\\
         0 & 0 & 0 & 1 & 0 & 1 & 0 & 1\\
         0 & 0 & 0 & 0 & 1 & 0 & 1 & 0\\
         0 & 0 & 0 & 0 & 0 & 1 & 1 & 0\\
         0 & 0 & 0 & 0 & 0 & 0 & 1 & 1\\
         0 & 0 & 0 & 0 & 0 & 0 & 0 & 1 \\
         \end{array}$$
         
         Therefore $y^{-1} = 10001010$
        \end{proof}
        
        
    \end{enumerate}
    
    \item Let a $(n,k)$-linear code $C \subseteq \mathbb{F}^n_q$ be given by its generator matrix $G$. Let $H$ be the generator matrix of the dual code. Show that $GH^{T} = 0_{k \times (n-k)}$.
    
    \emph{Remark: The check matrix can be defined either to be the generator matrix of the dual code $C^* = \{ x \in \mathbb{F}^n_q \: : \: x \cdot c = 0 \textrm{ for all }c \in C\}$ or to be a matrix that satisfies $GH^{T} = 0$. This exercise shows, that those two definitions are equivalent.}
    
    \begin{proof}
    
    Notice that $H$ is of dimension $(n-k) \times n$ and $G$ is of dimension $k \times n$. Therefore $G H^{T}$ is of dimension $k \times (n-k)$
    Consider $(GH^T)_{i,j} = \displaystyle \sum \limits_{k=1}^n G_{i,k} H_{j,k} = G_i \cdot H_j = 0$. Since $G_i \in C$ and $H_j \in C^*$.
    \end{proof}
    
    \item Compute the dual code $C^*$ of a linear code $C \subseteq \mathbb{F}^8_2$ that is represented by
    
    $$G= \left( \begin{array}{*{8}{c}}
         1 & 0 & 0 & 0 & 1 & 1 & 0 & 1  \\
         0 & 1 & 0 & 0 & 1 & 0 & 1 & 1 \\
         0 & 0 & 1 & 0 & 1 & 1 & 1 & 0 \\
         0 & 0 & 0 & 1 & 0 & 1 & 1 & 1 
         
    \end{array} 
    \right )$$
    
    \begin{proof}
    Recall that if a code is systematic, given by a generating matrix $G = [I_k \: \vert \: P]$ then the dual code has generating matrix $H = [ -P^T \: \vert \: I_k]$.
    
    Notice that $P$ is given by $$ \left ( \begin{array}{*{4}{c} }
         1 & 1 & 0 & 1 \\
         1 & 0 & 1 & 1 \\ 
         1 & 1 & 1 & 0 \\ 
         0 & 1 & 1 & 1 \\ 
    \end{array}\right) $$
    
    Since in the code is binary, then $-P = P$, therefore $-P^T$ is given by the following matrix
    
    $$\left ( \begin{array}{*{4}{c} }
         1 & 1 & 1 & 0 \\
         1 & 0 & 1 & 1 \\ 
         0 & 1 & 1 & 1 \\ 
         1 & 1 & 0 & 1 \\ 
    \end{array}\right) $$
    
    Therefore $H$ is given by 
    $$ H = \left ( \begin{array}{*{8}{c} }
         1 & 1 & 1 & 0 & 1 & 0 & 0 & 0\\
         1 & 0 & 1 & 1 & 0 & 1 & 0 & 0\\ 
         0 & 1 & 1 & 1 & 0 & 0 & 1 & 0\\ 
         1 & 1 & 0 & 1 & 0 & 0 & 0 & 1\\ 
    \end{array}\right) $$
    \end{proof}
    \item Consider a linear code $C \subseteq \mathbb{F}_2^5$ with generator matrix
    $$ \left ( \begin{array}{*{5} {c}}
         1 & 0 & 0 & 1 & 1  \\
         0 & 1 & 0 & 0 & 1 \\
         1 & 1 & 1 & 1 & 1
         
    \end{array} \right)$$
    
    \begin{proof}
    First we will make the generator matrix systematic through diagonalization.
        $$ G' = \left ( \begin{array}{*{5} {c}}
         1 & 0 & 0 & 1 & 1  \\
         0 & 1 & 0 & 0 & 1 \\
         0 & 0 & 1 & 0 & 1
         
    \end{array} \right)$$
    
    This yields the following matrix $H$ as a check matrix
    
$$H = \left ( \begin{array}{*{5} {c}}
         1 & 0 & 0 & 1 & 0  \\
         1 & 1 & 1 & 0 & 1 \\
         
    \end{array} \right)$$
    
    Notice that the code $C = \{ 00000, 10011, 01001, 11010, 00101, 10110, 01100, 11111 \}$ 
    
    $H^T = \left ( \begin{array}{cc}
        1 & 1  \\
        0 & 1  \\
        0 & 1  \\
        1 & 0  \\
        0 & 1  
    \end{array} \right )$
    
    Notice that the minimum distance of this code is $2$. Therefore up to $1$ errors can be detected and 1 can be corrected.
    
    The correcting scheme can be given as follows:
    
    If $S(v) \in \{01,10 \}$ then simply $\Bar{v} = v +S(v)$.
    
    If $S(v) = \{ 11\}$, then $C' = \{00011, 10000, 01010, 00110,10101, 01111,11100 \}$ 
    
    Since the distances can be at most 1, then the corrected code should be $\{00011 \mapsto 10011, 10000 \mapsto 00000,01010 \mapsto \textrm{error}, 00110 \mapsto 10110, 10101 \mapsto\textrm{error}, 01111 \mapsto 11111,11100 \mapsto 01100  \}$
    \end{proof}
    
    \item Let $p(x)= x^3+2$ be a generating polynomial of a cyclic $(9,6)$-linear code over $\mathbb{F}_3$. Determine a generating matrix such that the code is a systematic code, i.e. encoding is done by appending one or more letters at the end of the original words.
    
    \begin{proof}
    The generator matrix is given by:
    
    $$\left ( \begin{array}{*{9} c}
         2 & 0 & 0 & 1 & 0 & 0 & 0 & 0 & 0  \\
         0 & 2 & 0 & 0 & 1 & 0 & 0 & 0 & 0 \\ 
         0 & 0 & 2 & 0 & 0 & 1 & 0 & 0 & 0 \\
         0 & 0 & 0 & 2 & 0 & 0 & 1 & 0 & 0 \\
         0 & 0 & 0 & 0 & 2 & 0 & 0 & 1 & 0 \\
         0 & 0 & 0 & 0 & 0 & 2 & 0 & 0 & 1 \\
    \end{array} \right )$$
    
    Operating the matrix above through elementary operations yields the following systematic generator matrix:
    
    $$G =\left ( \begin{array}{*{9}c}
         1 & 0 & 0 & 0 & 0 & 0 & 2 & 0 & 0\\
         0 & 1 & 0 & 0 & 0 & 0 & 0 & 2 & 0\\
         0 & 0 & 1 & 0 & 0 & 0 & 0 & 0 & 2\\
         0 & 0 & 0 & 1 & 0 & 0 & 2 & 0 & 0 \\
         0 & 0 & 0 & 0 & 1 & 0 & 0 & 2 & 0\\
         0 & 0 & 0 & 0 & 0 & 1 & 0 & 0 & 2\\
    \end{array}\right)$$
    \end{proof}
    
    \item Let $(s_n)_{n \geq 0}$ be a homogeneous linear recurring sequence of order $k$ over a finite field $\mathbb{F}_q$
    
    $$ s_{n+k} = a_0 s_n + a_1 s_{n+1}+ \ldots + a_{k-1}s_{n+k-1},$$
    
    where $a_0, a_1, \ldots, a_{k-1} \in \mathbb{F}_q$ are fixed and $s_0, s_1, \ldots, s_{k-1}$ are given. Show that $(s_n)_{n \geq 0}$ has to be a periodic sequence.
    \emph{Hint: use linear feedback shift registers.}
    
    \begin{proof}
    Notice that this sequence is generated by a linear shift register with $k$ different registers, each of which has $q$ different possible states. Notice that there are at most $q^k$ different possible states for the linear shift register. This implies that the state of the registers should repeat in $q^k$ different states or less. Therefore the sequence generated is periodic.
    \end{proof}
\end{enumerate}


\end{document}