\documentclass[12pt]{article}
\usepackage{mathrsfs}
\usepackage{amsmath,amssymb}
\usepackage{graphicx}
\usepackage[utf8]{inputenc}
\usepackage{amsthm}
\usepackage{tkz-berge}
\usepackage{pgf}
\usepackage{xcolor}
\usepackage{algorithm}
\usepackage{caption}
\usepackage{multicol}
\usepackage{array}
\usepackage[noend]{algpseudocode}
\usepackage{listings}
\newcommand{\Term}{Winter 2019}
\newcommand{\Course}{104.272 Discrete Mathematics, Group 1}

\newcommand{\Assignment}{9. Exercise}
\newcommand{\DueDate}{ 11 December, 2019 }

\usepackage[body={6in,9in}]{geometry}



\begin{document}
Hugo \textit{Rincon Galeana}
\begin{center}

\textbf{TU Wien, \Term} \\
\textbf{\Course} (Professor Gittenberger) \\
\textbf{\Assignment, Due \DueDate}
\end{center}


%%%%%%%%%%%%%%%%%%%%%%%%%%%%%%%%%%%%%%%%%%%%%%%%%%%%%%%%%%%%%%%%%%%%%%%%
%% *****      Type your answers after the "\soln" commands      ***** %%
%%%%%%%%%%%%%%%%%%%%%%%%%%%%%%%%%%%%%%%%%%%%%%%%%%%%%%%%%%%%%%%%%%%%%%%%
\begin{enumerate}
    \setcounter{enumi}{80}
    \item Find (without using a computer) the last two digits of $2^{1000}$.
    \begin{proof}
    
    First notice that $\textrm{gcd}(2,25) = 1$, therefore, applying Euler's theorem yields $$ 2^{\varphi(25)} \underset{\textrm{mod } 25}{\equiv} 1$$.
    
    Notice also that $\varphi(25) = 25 \cdot \displaystyle \left (1- \frac{1}{5} \right) = 20$.
    
    Therefore $2^{20} \underset{\textrm{mod } 25}{\equiv} 1$; which implies that $2^{1000} \underset{\textrm{mod } 25}{\equiv} 1$.
    
    Now, notice that $2^{1000} \underset{\textrm{mod } 4}{\equiv} 0$.
    
    It follows from the Chinese Residue Theorem that there exists one unique solution modulo 100 for the following system of modulo equations.
    
    $$ x \underset{\textrm{mod }25}{\equiv} 1$$
    $$ x \underset{\textrm{mod }4}{\equiv} 0$$
    
    From the previous equation system it follows that $x = 25 \cdot z_1 + 1$ and $x = 4 \cdot z_2$. Notice that $4 \cdot z_2 - 25 \cdot z_1 = 1$. Notice that $4 \cdot (-6) - 25 (-1)$. This shows that $-24 = 76$ is the only solution for the equations  modulo 100. Notice that $2^{1000}$ is also a solution. Therefore $2^{1000} \underset{\textrm{mod }100}{\equiv} 76$. Therefore these are the last 2 numbers in decimal script of $2^{1000}$.
    \end{proof}
    
    \item Let $a$ and $b$ be two natural numbers such that $\textrm{gcd}(a,b)=1$. Prove that there exists a natural number $c$ with $ac \underset{\textrm{mod }b}{\equiv} 1$. Find such $c$ for $a=55$ and $b = 42$.
    
    \begin{proof}
    Since $\textrm{gcd}(a,b)=1$, there is a linear combination $\alpha \cdot a + \beta \cdot b =1$ such that $\alpha ,\beta \in \mathbb Z$. Therefore $\alpha \cdot a = 1 +(-\beta) \cdot b$. This is the definition of $\alpha \cdot a \underset{\textrm{mod }b}{\equiv} 1$. Let $c:= \alpha$.
    
    For $a = 55$ and $b = 42$, we apply the Euclidian algorithm to express 1 as a linear combination of $55$ and $42$. We get that $1 = 13 \cdot 55 - 17 \cdot 42$. Therefore $c:= 13$.
    \end{proof}
    \pagebreak
    
    \item  Let $a$ and $b$ be two natural numbers. Prove or disprove:
        \begin{enumerate}
            \item If $\textrm{gcd}(a,b) = 1$ then $\textrm{gcd}(a^2,ab,b^2) = 1$.
            \begin{proof}
                Notice that since $\textrm{gcd}(a,b)=1$ then $(a^2,ab) = a$ from using the prime power factorization of $a$ and $b$. From the prime power factorization of $a$ and $b$ $\textrm{gcd}(a,b^2) = 1$. This shows that $\textrm{gcd}(a^2, ab , b^2)$.
            \end{proof}
            
            \item If $a^2 | b^3$ then $a|b$.
            
            FALSE
            \begin{proof}
            Consider $a = 2^3; b = 2^2$, $a \not | \;b$ but $a^3 = 2^6 \; | \; b^3 = 2^6$.
            \end{proof}
            
        \end{enumerate}
        
    \item Prove that if a prime number $p$ satisfies $\textrm{gcd}(a,p-1)=1$, then for every integer $b$ the congruence relation $x^a \underset{\textrm{mod }p}{\equiv} b$ admits a solution.
    
    \begin{proof}
        Notice that since $p$ is a prime, then $\varphi(p) = p-1$. It follows from Euler's Theorem that for all $u \neq 0 \in \mathbb Z_p$, $u^{p-1} = 1$. 
        
        Now notice that $0$ trivially satisfies $x^a \underset{\textrm{mod} p}{\equiv} 0$.
        
        Now consider the equation $x^n \underset{\textrm{mod }p}{\equiv} u$ for some $u \underset{\textrm{mod } p}{\not\equiv} 0$.
        
        Notice that $u$ is a unit in $\mathbb Z_p$, therefore $u^{z}$ is well defined for all $z \in \mathbb Z_p$. Since $\textrm{gcd}(a,p-1) = 1$ there exists $\alpha, \beta \in \mathbb Z$ such that $\alpha \cdot (p-1) + \beta \cdot a = 1 $. It follows that in $\mathbb Z_p$: 
        $$ u^{\alpha \cdot (p-1) + \beta \cdot a}= u$$
        $$ (u^{(p-1)})^{\alpha} \cdot (u^{\beta})^a = u$$
        $$ (u^{\beta})^a = u$$
        
        Therefore $u^\beta$ is a solution for equation $x^n \underset{\textrm{mod }p}{\equiv} u$.
    \end{proof}
    
    \item Use the Chinese remainder theorem to solve the following system of congruence relations
    $$ 3x \underset{\textrm{mod }13}{\equiv}12$$
    $$ 5x \underset{\textrm{mod }22}{\equiv}7$$
    $$ 4x \underset{\textrm{mod }14}{\equiv}6$$
    
    \begin{proof}
       Notice that this system of equations can be reduced to
       
       $$ x \underset{\textrm{mod }13}{\equiv} 4 $$
       $$ x \underset{\textrm{mod }22}{\equiv} -3$$
       $$ x \underset{\textrm{mod }7}{\equiv} 5$$
       
       Since both 3 and 5 are units modulo 13 and 22 respectively. Notice that $4x \underset{\textrm{mod }14}{\equiv}6$ is equivalent to $4 \cdot x + \alpha \cdot 14 = 6$ for some $\alpha \in \mathbb Z$. Therefore $ 2 \cdot x + \alpha \cdot 7 = 3$, this is equivalent to $2 \cdot x \underset{\textrm{mod }7}{\equiv}3 $ which in turn is equivalent to $x \underset{\textrm{mod }7}{\equiv} 5$.
       
       Now we can proceed to apply the Chinese Remainder Theorem.
       
       We need to find an $x_1$ that solves $ 154 \cdot x_1 \underset{\textrm{mod }13}{\equiv} 4$, which is equivalent to $-2 \cdot x_1 \underset{\textrm{mod }13}{\equiv} 4$. Therefore $x_1 = -2$.
       
       Now we proceed to find a solution for $91 \cdot x_2 \underset{\textrm{mod }22}{\equiv} -3$. This is equivalent to finding a solution for $3 \cdot x_2 \underset{\textrm{mod }22}{\equiv} -3$. Notice that $x_2 = -1$ is a solution.
       
       At last we find a solution for $286 \cdot x_3 \underset{\textrm{mod }7}{\equiv} 5$ which is equivalent to finding a solution for $6 \cdot x_3 \underset{\textrm{mod }7}{\equiv} 5$ which in turn is equivalent to $-1 \cdot x_3 \underset{\textrm{mod }7}{\equiv} -2$. Therefore $x_3=2$ is a solution.
       
       We proceed to build the global solution by considering $7\cdot 22 \cdot -2 + 13 \cdot 7 \cdot -1 + 13 \cdot 22 \cdot 2 = 173$
    \end{proof}
    
    In the next three exercises $\lambda$ will denote the Carmichael function and $\varphi$ Euler's totient function.
    
    \item Compute $\lambda(49392)$ and $\varphi(49392)$
    
    \begin{proof}
        We begin by obtaining the prime factorization of $z := 49392 = 2^4 \cdot 3^2 \cdot 7^3 $ via the Sieve of Eratostenes.
        
        Since $\varphi (a,b) = \varphi(a) \cdot \varphi(b)$ for relatively prime $a,b$. Then $\varphi(z) = \varphi(2^4) \cdot \varphi (3^2) \cdot \varphi(7^3)$.
        
        Recall that 
        $$ \varphi(p^r) = p^{r-1}(p-1) $$
        
        Therefore $\varphi(2^4) = 2^3 $, $\varphi(3^2) = 3 \cdot 2$, $\varphi (7^3) = 7^2 \cdot 2 \cdot 3$.
        
        It follows that $\varphi(z) = 2^5 \cdot 3^2 \cdot 7^2 $.
        
        Notice that $\lambda(z) = \textrm{lcm}[\lambda(2^4), \lambda(3^2), \lambda(7^3)]$.
        
        Recall that $$\lambda (1) = 1; \lambda(2) = 1; \lambda(4) = 2$$  $$ \lambda (2^e) = 2^{e-2} \textrm{ for } e \geq 3$$  $$ \lambda(p^e)= p^{e-1}(p-1) \textrm{ for } p \in \mathbb P; p \neq 2$$
        
        Therefore $\lambda(2^4) = 2^2 $; $\lambda(3^2)= 3 \cdot 2$; $\lambda(7^3) = 7^2 \cdot 3 \cdot 2$.
        
        $\lambda(z) = \textrm{lcm}[2^2, 2 \cdot 3, 2 \cdot 3 \cdot 7^2] = 2^2 \cdot 3 \cdot 7^2$
        
        $\varphi(z) = 2^5 \cdot 3^2 \cdot 7^2 $; $\lambda(z) = 2^2 \cdot 3 \cdot 7^2$
    \end{proof}
    
    \item Prove that for all $m,n \in \mathbb N ^+$, the following identity holds:
    $$ \varphi(m \cdot n)= \varphi(m) \varphi(n)\displaystyle\frac{\textrm{gcd}(m,n)}{\varphi(\textrm{gcd}(m,n))}$$.
    
    \begin{proof}
    Let us recall that if $n = p_1^{a_1}p_2^{a_2} \ldots p_k^{a_k}$ is the prime power factorization of $n$, then $\varphi(n) = n \displaystyle \left(1- \frac{1}{p_1}\right) \ldots \left(1- \frac{1}{p_k}\right)$.
    
    Let $p_1, \ldots , p_r$ be the prime divisors of $m$ that don't divide $n$, $q_1, \ldots, q_s$  the prime divisors of $n$ that don't divide $m$, and $r_1, \ldots, r_t$, the common prime divisors of $m$ and $n$.
    
    Let $$P := \displaystyle\prod\limits_{i=1}^{r}\left( 1- \frac{1}{p_i}\right)$$
    $$Q := \displaystyle\prod\limits_{i=1}^{s}\left( 1- \frac{1}{q_i}\right)$$
    $$R := \displaystyle\prod\limits_{i=1}^{t}\left( 1- \frac{1}{r_i}\right)$$
    
    It follows that $\varphi (m) = m \cdot P \cdot R$, $\varphi(n) = n \cdot Q \cdot R$, and $\varphi (m\cdot n) = m \cdot n \cdot P \cdot Q \cdot R = \displaystyle\frac{m \cdot Q \cdot R \cdot n \cdot Q \cdot R}{R} = \displaystyle\frac{\varphi(m) \cdot \varphi(n)}{R}$.
    
    Since $\textrm{gcd}(m,n)$ is a common divisor, then the prime power factorization of $\textrm{gcd}(m,n)$ is given by $r_1,\ldots, r_t$. it follows that $\varphi(\textrm{gcd}(m,n))= \textrm{gcd}(m,n) \cdot R$. Therefore $R = \displaystyle\frac{ \varphi( \textrm{gcd}(m,n))}{\textrm{gcd}(m,n)}$
    
    Therefore $\varphi(m \cdot n ) = \varphi(m) \cdot \varphi(n) \cdot \displaystyle\frac{\textrm{gcd}(m,n)}{\varphi(\textrm{gcd}(m,n))}$.
    \end{proof}
    
    \item Show that $m|n$ implies $\lambda(m) | \lambda (n)$.
    
    \emph{Hint: first prove that}
    $$a_i | b_i \textrm{ for } i = 1, \ldots,k \Longrightarrow \textrm{lcm} (a_1,a_2, \ldots, a_k)\; | \textrm{ lcm}(b_1, b_2, \ldots,b_k) $$.
    
    \begin{proof}
        We will first prove the hint.
        
        Consider the $S = \{p_1, \ldots , p_m\}$ the set of primes that divide some $b_i$ with $i \in \{1,\ldots,k\}$. Let $a_{i,j}$ be the power of $p_j$ in the prime power factorization of $a_i$, and $a_{i,j}$ be defined in the same way for $b_i$.
        
        Notice that the $\textrm{lcm}(b_1, \ldots, b_k) = \displaystyle\prod \limits_{j=1}^{m} p_i^{max_{i}{b_{i,j}}}$.
        
        Likewise $\textrm{lcm}(a_1, \ldots, a_k) = \displaystyle\prod \limits_{j=1}^{m} p_i^{max_{i}{a_{i,j}}}$.
        
        Notice that since each $a_i | b_i$, then for each $i$, $a_{i,j} \leq b_{i,j}$. Therefore $max_i{a_{i,j}}\leq max_i{b_{i,j}}$.
        
        Therefore $\textrm{lcm}(a_1, \ldots, a_k) \; | \; \textrm{lcm}(b_1, \ldots, b_k)$
        
        Now consider $m = p_1^{m_1} \ldots p_k^{m_k}$ and $n = p_1^{n_1}\ldots p_k^{n_k}$
        
        Since $m \;|\; n$ it follows that each $m_i \leq n_i$.
        
        Recall that $\lambda\left ( \prod\limits_{i=1}^{k} p_i^{e_i} \right) = \textrm{lcm} (\lambda(p_1^{e_1}),\ldots, \lambda(p_k^{e_k}))$
        
        Therefore $\lambda(m) = \textrm{lcm}(\lambda(p_1^{m_1}),\ldots, \lambda(p_k^{m_k}))$ and $\lambda(n) = \textrm{lcm}(\lambda(p_1^{n_1}),\ldots, \lambda(p_k^{n_k}))$
        
         Also recall that $$\lambda (1) = 1; \lambda(2) = 1; \lambda(4) = 2$$  $$ \lambda (2^e) = 2^{e-2} \textrm{ for } e \geq 3$$  $$ \lambda(p^e)= p^{e-1}(p-1) \textrm{ for } p \in \mathbb P; p \neq 2$$
        
        Also since each $m_i \leq n_i$, then it follows that $\lambda(p_i^{m_i}) \; | \; \lambda(p_i^{n_i})$.
        
        Applying the proof of the hint yields that 
        $$ \lambda(m) = \textrm{lcm}(\lambda(p_1^{m_1}),\ldots, \lambda(p_k^{m_k})) \; | \; \lambda(n) = \textrm{lcm}(\lambda(p_1^{n_1}),\ldots, \lambda(p_k^{n_k})) $$
    \end{proof}
    
    \item Let $(n,e) =(3233,49)$ be a public RSA key. Compute the associated decryption key $d$.
    
    \begin{proof}
    The key is noting that $3233 = 53 \cdot 61$. Let's consider $\textrm{lcm}(52 = 2^2 \cdot 13, 60 = 2^2\cdot 3 \cdot 5) = 2^2 \cdot 3 \cdot 5 \cdot 13 = 780$.
    
    In order to find the decription key $d$, we apply the Euclidian algorithm to find a linear combination of $780 $ and $49$ that is equal to 1. 
    
    This leads to $$ 12 \cdot 780 -191 \cdot 49 = 1$$.
    
    Therefore the public key $d = -191 = 589$. A quick sanity check verifies that indeed $589 \cdot 49 \underset{\textrm{mod }780}{\equiv} 1$.
    \end{proof}
    
    \item Consider the encoding of a string $s$, parsed into blocks of two letters, via the mapping 
    $$ A \mapsto 01, \quad B \mapsto 02, \quad \ldots , \quad Z \mapsto 26$$.
    
    Thus $s$ is encoded into a sequence of integers, one for each block and each with at most four digits. For example, $s= \textrm{BAZC} \mapsto (201,2603)$. Each element of the list is then further encoded using the public RSA key of exercise 89.
    
    The sequence $(2701,2593,371,1002)$ was encoded via the two steps described above.
    
    Decode it, i.e. find the original string.
    
    For this purpose we help ourselves with the following python snippet.
    
    \begin{verbatim}
    def modp (n,m,k):
        ans = 1
        for i in range(m):
	            ans = (ans*n)%k
	        return ans

    def letter (n):
        print chr(n+96)
    \end{verbatim}
    
    Since we already cracked the key $d$ in the previous exercise, then we only need to input the blocks and raise them to the $589^{\textrm{th}}$ power modulo $3233$.
    
    This yields the following blocks $(0315,1316,2120,0518)$ which is decifered as 'computer'.
    
    
\end{enumerate}
\end{document}